\documentclass{kluwer}    % Specifies the document style.

\newdisplay{guess}{Conjecture}

% package provides \sout{} strikeout macro -- remove this before submission
\usepackage{ulem}

\begin{document}
\begin{article}
\begin{opening}         
\title{Expected Dependency Pair Match:\\
Predicting {HTER} improvements with expected syntactic structure} 
\author{Jeremy G. \surname{Kahn}\email{jgk@u.washington.edu}}  
\institute{University of Washington}
\author{Matthew \surname{Snover}\email{snover@cs.umd.edu}}
\institute{University of Maryland}
\author{Mari \surname{Ostendorf}\email{ostendor@u.washington.edu}}  
\institute{University of Washington}
\runningauthor{Kahn, Snover \& Ostendorf}
\runningtitle{Expected Dependency Pair Match}
%s\date{May 10, 2009}

\begin{abstract}
Abstract to go here
\end{abstract}
\keywords{machine translation evaluation, syntax, dependency trees}

\end{opening}           

% override ulem 's default change of \emph{} -- remove before ship
\normalem
\section{Introduction}

\begin{itemize}
\item \sout{ evaluation [esp. automatic evaluation]}
\item \sout{popular current
  measures BLEU \cite{papineni02bleu}, TER}
\end{itemize}
Machine translation (MT) evaluation is a challenge for research
because the space of good translations is large, and two equally good
translations may appear to be quite different at first glance. 
%
The challenges of choosing among translations are compounded when this
evaluation is done automatically.
%
Human evaluation, however, is both time-consuming and difficult, so
research has turned increasingly towards automatic measures of
translation quality, usually by comparing the system translation to
one or more reference (human) translations.
%
Automatic measures of this kind (e.g.\ BLEU \cite{papineni02bleu}) not
only provide a well-defined evaluation standard but are also required
for training on error criteria, e.g.\ with minimum error rate training
\cite{och03mert}.

The most popular evaluation measures are currently BLEU (based on
$n$-gram precision) and the edit-distance measure Translation Edit
Rate (TER) \cite{snover06ter}.  Recent research has found that these
measures may not accurately track translation quality both empirically
\cite{charniak03syntaxlmmt} and theoretically
\cite{callisonburch06bleuproblems}.

\begin{itemize}
\item \sout{challenges - motivate search for better measures}
\item \sout{other extensions:}
  \begin{itemize}
  \item \sout{METEOR }
    \begin{itemize}
    \item \sout{stemming and synonym tables}
    \item \sout{tuned but not generally     tuneable}
    \end{itemize}
  \item \sout{TERp}
    \begin{itemize}
    \item \sout{development-set tuning by user}
    \item \sout{stemming and synonym tables}
    \item \sout{paraphrase tables}
    \end{itemize}
  \end{itemize}
\end{itemize}

These challenges have motivated a search for better measures that
incorporate additional language knowledge sources.  METEOR
\cite{banerjee05meteor}, for example, uses synonym tables and
morphological stemming to do progressively more-forgiving matching.
It can be tuned towards recall or precision, but is generally not
tuned by users.  TERp \cite{snover09terp} is an extension of the
previously-mentioned TER that also incorporates synonym sets, along
with automatically-derived paraphrase tables.  TERp is explicitly
intended to be tuned to a development set by users.


\begin{itemize}
\item \sout{prior work with (partial) syntactic locality:}
  \begin{itemize}
  \item \sout{Liu \& Gildea }
  \item \sout{Roark et al. Sparseval}
  \item \sout{Owczarzak et al. 2007}
  \end{itemize}
\end{itemize}

As an alternative to these synonym- and paraphrase-based approaches,
other metrics model syntactically-local (rather than string-local)
word-sequences. \cite{liu05syntaxformteval} compared tree-local
$n$-gram precision in various configurations of constituency and
dependency trees.  The dependent-based SParseval measure
\cite{roark06:sparseval}, originally designed as a parse-quality
metric for speech, is a similar approach, in that it is an F-measure
over a decomposition of reference and hypothesis trees.
\cite{owczarzak07labelleddepseval}'s \textbf{d} and \textbf{d\_var}
measures compare LFG-derived relational tuples from reference and
hypothesis translations and report substantial improvement in
correlation with human judgment relative to BLEU and TER.

These syntactically-oriented measures require a system for proposing
dependency structure over the reference and hypothesis
translations. Some \cite{liu05syntaxformteval,roark06:sparseval} use
PCFG parsers with deterministic head-finding, while others
\cite{owczarzak07labelleddepseval}extract the semantic dependency
relations from an LFG parser \cite{cahill04lfg}.
%
This work extends the dependency-scoring strategies of
\cite{roark06:sparseval,owczarzak07labelleddepseval} using a
widely-used and publically available PCFG parser and deterministic
head-finding rules.
 

\begin{itemize}
\item Evaluation of evaluation measures
  \begin{itemize}
  \item correlation vs. human fluency/adequacy 
  \item correlation vs. human-in-the-loop measures (HTER)
  \end{itemize}
\item one challenge of metric-evaluation: underlying doc difficulty 
  \begin{itemize}
  \item describe mean-subtraction
  \end{itemize}
\end{itemize}

\section{Experimental paradigm}

Section explains space of measures to explore

\begin{itemize}
\item F-measure over various bags of components of parse tree

  [ Figure describing Sparseval-ish variant ]

\item Listing possible component valences
  \begin{itemize}
  \item dependent + labeled link + head 
  \item outbound only (dependent +
    label)
  \item inbound only (label + head)
  \item word-word (skipping label)
  \end{itemize}
  \begin{itemize}
  \item 1-grams
  \item 2-grams
  \item \ldots{}
  \end{itemize}
  
  Note: inbound links \& full pairs "overweight" head words; this is
  a good thing.

  [ figure demonstrating F-measure over d+l+h decomposition of toy
  trees ]
    
\item Tree features may be extracted from a 1-best tree, or from an
  n-best tree list 

  \begin{itemize}
  \item  counts are an expectation over the forest
  \item n parameter: how many trees in the forest
  \item gamma parameter for flattening overconfident parse hyps
  \end{itemize}
  note intentional similarities to Owczarzak paper


\item Components in the basic DPM measures combined in various ways:

  \begin{itemize}
  \item combination of precision, recall scores for different
    components (on analogy with BLEU combination), naive assumption:
    all weighted equally

  \item simple F over items of all valences (naive assumption: all
    weighted equally)
  \end{itemize}
\item Tuneable weights:
  \begin{itemize}
  \item recall that TERp has tuneable system over small number of parameters
    \begin{itemize}
    \item insertion weight
    \item deletion weight
    \item substitution weight
    \item synonym weight (motivated with WordNet)
    \end{itemize}
  \item learn weights for separate precision, recall for various
    components
  \end{itemize}
  
\item Implementation
  \begin{itemize}
  \item Charniak parser
  \item n=1..50
  \item head-finding, with some modifications
  \item label-construction
  \end{itemize}
\end{itemize}

\section{Correlation with human judgments of fluency \& adequacy}

Corpus: LDC Multiple Translation Chinese corpus 3\&4 treat each
segment and each judgment as independent data point

\begin{itemize}
\item EXPT

  using n=1 parse for DPM variants

\begin{verbatim}
   metric      r
  --------   -----
  F[dl;lh]   0.226
  BLEU4      0.218  # +1 smoothing
  F[dlh]     0.185
  TER       -0.173
\end{verbatim}

  point: using partial-labeling (dl,lh) much better than full
  link. Also, using Charniak parser as labeler seems to work, LFG not
  necessary

\item EXPT 

  combining precision, recall naively vs. combining items

\begin{verbatim}
  F(1g,2g,dl,lh)        0.237
  prmeans(1g,2g,dl,lh)  0.217

  F(dl,lh)              0.226
  prmeans(dl,lh)        0.208
\end{verbatim}

  point: combining individual items naively (before computing
  precision and recall) seems better than naively combining precisions
  and recalls (too many chances for zeroes?)

\item EXPT
  expanding n, gamma=0

\begin{verbatim}
  F[1g,2g,dl,lh],n=50   0.239
    ...          n=1    0.237
  F[dl,lh],n=50         0.234
    ...    n=1          0.226
\end{verbatim}
  point: larger n $\to$ better r
  (holds for other comparisons too, but these are good examples)

\item  EXPT

  setting gamma

  Tuning experiments finds that gamma of 0.25 is slightly better
  (especially when using F[dl,lh]) than gammas of other values in
  range 0-1 (these values probably not significant)

\end{itemize}

\section{Correlations with HTER}

  Corpus
    Gale translation corpus 2.5; 3 sites, 2 source languages, 4 genres each

  prediction:
    Mean-subtracted HTER per-document

  base measure EDPM: F[1g,2g,dl,lh],n=50,gamma=0.25

  [EXPT]
\begin{verbatim}
            Ar     Zh    All
   TER     0.51   0.32  0.44
   BLEU_4 -0.42  -0.33 -0.37
   EDPM   -0.60* -0.39 -0.50*
\end{verbatim}
  point: base EDPM does better on all docs, and on Arabic. difference
  is not quite significant at the p=0.05 on the Chinese documents

  Note 1: also tried pairwise deltas per-document, with similar
  results, as presented in the MetricsMATR competition submission.

  Note 2: when using per-segment mean-subtracted, r values are smaller
  and *Arabic* was the language where EDPM did not perform best.

\begin{verbatim}
               Ar     Zh     All
      TER     0.38*  0.04   0.19
      BLEU_4 -0.10  -0.16  -0.14
      EDPM   -0.31  -0.19* -0.24*
\end{verbatim}

  (Go into details of per-genre differences here? or leave out, for
  space reasons? -- leave out, I'd say)

\section{Weight-tuning to combine syntax and other knowledge sources}

TERp optimizer [Matt, a brief description here]

TERp comes with a variety of features (described earlier)

include optionally new features

Corpus: GALE 2.5 data, again documents randomly split into two groups,
evenly split across language and genre

tuning target: weighted correlation with mean-removed segment HTER

[we use weighted correlation to avoid emphasis on short segments
(which hurts document-level and system-level utility) ]

Select features from set:

\begin{description}
\item[E]: features from EDPM, specifically, P,R for inbound, outbound,
  full-links, and unlabeled links  (8 features)
\item[T]: features from basic TERp
  (inserts, deletes, substitutions, shifts)  (7 features)
\item[P]: features from TERp paraphraser
  (4 features)
\item[N]: precision and recall for 1-grams, 2-grams
\item[B]: brevity/prolixity (2 features: one for longer-than-ref, one for
  shorter-than-ref)
\end{description}

tune weights on one set, test on the other (and vice versa).
reporting average r between the two.

\begin{verbatim}
   == Tuned For Weighted Mean Removed Pearson
     - Examining WGT_MEAN_RM_SEG_PEAR ==
            Average
      Cond   Test  
     ---------------
      ETPB   0.4096
      ETP    0.4038
      TPB    0.4010
      TP     0.3946
      TPNB   0.3940
      TPN    0.3891
      ETB    0.3871
      ET     0.3827
      TB     0.3815
      ETNB   0.3814
      ETN    0.3804
      T      0.3758
      TNB    0.3708
      TN     0.3680
      E      0.3287
      EB     0.3267
      ENB    0.3149
      NB     0.3058
      EN     0.2973
      N      0.2636
      B      0.1638
\end{verbatim}
(maybe don't report all these numbers!)
  
point: E+T $>$ T, E+TB $>$ TB, E+TP $>$ TP; information is available in
syntax that is not captured in the other measures.

point: syntax is *not* just an expensive way to get at n-grams;
N-grams are not the same -- TP $>$ TP+N, but TP $<$ TP+E

\section{Conclusion}

expected dependency pair matching gets at something new

troubling area: speed of analysis -- current implementation requires
running a complete n-best parser

possible use: as a late-pass evaluation, to identify how systems
perform overall

future work:  explore ways to get at syntactic information without
the expense of a full parser:
\begin{itemize}
\item this work moves from an LFG parser (Owczarzak) to the
  substantially simpler and more robust Charniak system - keep going
  and look at simpler partial-parsing approaches (forests?)
\item conversely: how important is it that the parses be of
  high-quality?
\end{itemize}





\appendix

[don't think we want an appendix]


\acknowledgements

[grant numbers for the DARPA/GALE grants here]

% The endnotes section will be placed here.  But I don't think we have any

% \theendnotes

\bibliographystyle{klunamed}
\bibliography{edpm-paper}

\end{article}
\end{document}
