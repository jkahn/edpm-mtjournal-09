\documentclass{kluwer}    % Specifies the document style.

\newdisplay{guess}{Conjecture}

\begin{document}                                                                                   
\begin{article}
\begin{opening}         
\title{A Sample Document\thanks{Footnote 
            to the title with the `thanks' command.}} 
\author{Leslie \surname{Lamport}}  
\runningauthor{Leslie Lamport}
\runningtitle{A Sample Document}
\institute{Author's Affiliation}
\date{April 15, 1993}

\begin{abstract}
This is a sample input file.  Comparing it with the output it
generates can show you how to produce a simple document of
your own.
\end{abstract}
\keywords{sample, \LaTeX}


\end{opening}           

\section{Ordinary Text}  
                    % Produces section heading.  Lower-level
                    % sections are begun with similar 
                    % \subsection and \subsubsection commands.

The ends  of words and sentences are marked 
  by   spaces. It  doesn't matter how many 
spaces    you type; one is as good as 100.  The
end of   a line counts as a space.

One   or more   blank lines denote the  end 
of  a paragraph.  

Since any number of consecutive spaces are treated like a single
one, the formatting of the input file makes no difference to
      \TeX,         % The \TeX command generates the TeX logo.
but it makes a difference to you.  
When you use
      \LaTeX,       % The \LaTeX command generates the LaTeX logo.
making your input file as easy to read as possible
will be a great help as you write your document and when you
change it.  This sample file shows how you can add comments to
your own input file.

Because printing is different from typewriting, there are a 
number of things that you have to do differently when preparing 
an input file than if you were just typing the document directly.  
Quotation marks like 
       ``this'' 
have to be handled specially, as do quotes within quotes: 
       ``\,`this'                  % \, separates the double and single quote.
    is what I just 
    wrote, not  `that'\,''.  

Dashes come in three sizes: an 
       intra-word 
dash, a medium dash for number ranges like 
       1--2, 
and a punctuation 
       dash---like 
this.

A sentence-ending space should be larger than the space between words
within a sentence.  You sometimes have to type special commands in
conjunction with punctuation characters to get this right, as in the
following sentence.
       Gnats, gnus, etc.\    % `\ ' makes an inter-word space.
       all begin with G\@.   % \@ marks end-of-sentence punctuation.
You should check the spaces after periods when reading your output to
make sure you haven't forgotten any special cases.
Generating an ellipsis 
       \ldots\    % `\ ' needed because TeX ignores spaces after 
          % command names like \ldots made from \ + letters.
          %
          % Note how a `%' character causes TeX to ignore the 
          % end of the input line, so these blank lines do not
          % start a new paragraph.
with the right spacing around the periods 
requires a special  command.  

\TeX\ interprets some common characters as commands, so you must type
special commands to generate them.  These characters include the
following: 
       \$ \& \% \# \{ and~\}.

In printing, text is emphasized by using an
       {\em italic\/}  % The \/ command produces the tiny extra space that
               % should be added between a slanted and a following
               % unslanted letter.
type style.  

\begin{em}
   A long segment of text can also be emphasized in this way.  Text within
   such a segment given additional emphasis 
      with\/ {\em Roman} 
   type.  Italic type loses its ability to emphasize and become simply
   distracting when used excessively.  
\end{em}

It is sometimes necessary to prevent \TeX\ from breaking a line where
it might otherwise do so.  This may be at a space, as between the
``Mr.'' and ``Jones'' in
       ``Mr.~Jones'',        % ~ produces an unbreakable interword space.
or within a word---especially when the word is a symbol like
       \mbox{\em itemnum\/} 
that makes little sense when hyphenated across 
       lines.





\TeX\ is good at typesetting mathematical formulas like
       \( x-3y = 7 \) 
or
       \( a_{1} > x^{2n} / y^{2n} > x' \).
Remember that a letter like
       $x$        % $ ... $  and  \( ... \)  are equivalent
is a formula when it denotes a mathematical symbol, and should
be treated as one.


\section{Notes}
Footnotes\footnote{This is an example of a footnote.}
pose no problem\footnote{And another one}.
Endnotes\endnote{This is an example of an endnote.} pose no 
problem either\endnote{And another one.}.

When using the {\sc kluwer} style file, you can produce endnotes 
analogous to footnotes. Instead of the \LaTeX\ command \verb+\footnote{}+,
you can use \verb+\endnote{}+. The command \verb+\theendnotes+ can be used
to place the endnotes in the text. They will be put in a separate section in
the \verb+\footnotesize+ font.



\section{Displayed Text}

Text is displayed by indenting it from the left margin.
Quotations are commonly displayed.  There are short quotations
\begin{quote}
   This is a short a quotation.  It consists of a 
   single paragraph of text.  There is no paragraph
   indentation.
\end{quote}
and longer ones.
\begin{quotation}
   This is a longer quotation.  It consists of two paragraphs
   of text.  The beginning of each paragraph is indicated
   by an extra indentation.

   This is the second paragraph of the quotation.  It is just
   as dull as the first paragraph.
\end{quotation}
Another frequently-displayed structure is a list.
The following is an example of an {\em itemized} list, four levels deep.
\begin{itemize}
\item  This is the first item of an itemized list.  Each item 
      in the list is marked with a ``tick''.  The document
      style determines what kind of tick mark is used.
\item  This is the second item of the list.  It contains another
      list nested inside it.  The three inner lists are an {\em itemized}
      list.
    \begin{itemize}
       \item This is the first item of an enumerated list that
            is nested within the itemized list.
          \item This is the second item of the inner list.  \LaTeX\
            allows you to nest lists deeper than you really should.
      \end{itemize}
      This is the rest of the second item of the outer list.  It
      is no more interesting than any other part of the item.
   \item  This is the third item of the list.
\end{itemize}


The following is an example of an {\em enumerated} list, four levels deep.
\begin{enumerate}
\item  This is the first item of an enumerated list.  Each item 
      in the list is marked with a ``tick''.  The document
      style determines what kind of tick mark is used.
\item  This is the second item of the list.  It contains another
      list nested inside it.  The three inner lists are an {\em enumerated}
      list.
    \begin{enumerate}
       \item This is the first item of an enumerated list that
            is nested within the enumerated list.
          \item This is the second item of the inner list.  \LaTeX\
            allows you to nest lists deeper than you really should.
      \end{enumerate}
      This is the rest of the second item of the outer list.  It
      is no more interesting than any other part of the item.
   \item  This is the third item of the list.
\end{enumerate}


The following is an example of a {\em description} list.
\begin{description}
\item[Cow] Highly intelligent animal that can produce milk out of grass.
\item[Horse] Stupid animal that has legs.
\item[Human being] Stupid animal that can talk and think.
\end{description}

You can even display poetry.
\begin{verse}
   There is an environment for verse \\    % The \\ command separates lines
   Whose features some poets will curse.   % within a stanza.

               % One or more blank lines separate stanzas.

   For instead of making\\
   Them do {\em all\/} line breaking, \\
   It allows them to put too many words on a line when they'd 
   rather be forced to be terse.
\end{verse}

Mathematical formulas may also be displayed.  A displayed formula is
one-line long; multiline formulas require special formatting
instructions.
   \[  x' + y^{2} = z_{i}^{2}\]
Don't start a paragraph with a displayed equation, nor make
one a paragraph by itself.

Example of a theorem:


\begin{guess}
All conjectures are interesting, but some conjectures are more
interesting than others.
\end{guess}


\section{Tables and Figures}
Cross reference to labelled table: As you can see in Table~\ref{sphericcase} on 
page~\pageref{sphericcase} and also in Table~\ref{parset} on page~\pageref{parset}.





\begin{table} % 
\begin{tabular}{lrl}                                        
\hline
$Q_{s,\max}$   & [g/g DM h]  & 0.18\\
$K_{s}$       & [g/L]        & 1.0\\
$Y_{x/s}$     & [g DM/g]     & 0.5\\
$Y_{p/s}$     & [g/g]        & 0.854\\
$Q_{p,\max}$   & [g/g DM h]  & 0.0045\\
$\mu_{\rm crit}$  & [h$^{-1}$]  & 0.01\\
$k_{h}$       & [h$^{-1}$]  & 0.002\\
$m_{s}$       & [g/g DM h]  & 0.025\\                              
\hline
\end{tabular}
\caption[]{Parameter set used in the model of \shortcite{Bunt}}\label{parset}
\end{table}

\begin{table*}
\caption[]{The spherical case ($I_1=0$, $I_2=0$).}
\label{sphericcase}
\begin{tabular}{crrrrc}
\hline
Equil. \\
Points & \multicolumn{1}{c}{$x$} & \multicolumn{1}{c}{$y$} & \multicolumn{1}{c}{$z$} & \multicolumn{1}{c}{$C$} &
S \\
\hline
$~~L_1$ & $-$2.485252241 & 0.000000000 & 0.017100631 & 8.230711648 & U \\
$~~L_2$ &    0.000000000 & 0.000000000 & 3.068883732 & 0.000000000 & S \\
$~~L_3$ &    0.009869059 & 0.000000000 & 4.756386544 & $-$0.000057922 & U \\
$~~L_4$ &    0.210589855 & 0.000000000 & $-$0.007021459 & 9.440510897 & U \\
$~~L_5$ &    0.455926604 & 0.000000000 & $-$0.212446624 & 7.586126667 & U \\
$~~L_6$ &    0.667031314 & 0.000000000 & 0.529879957 & 3.497660052 & U \\
$~~L_7$ &    2.164386674 & 0.000000000 & $-$0.169308438 & 6.866562449 & U \\
$~~L_8$ &    0.560414471 & 0.421735658 & $-$0.093667445 & 9.241525367 & U \\
$~~L_9$ &    0.560414471 & $-$0.421735658 & $-$0.093667445 & 9.241525367 & U
\\
$~~L_{10}$ & 1.472523232 & 1.393484549 & $-$0.083801333 & 6.733436505 & U \\
$~~L_{11}$ & 1.472523232 & $-$1.393484549 & $-$0.083801333 & 6.733436505 & U
\\ \hline
\end{tabular}
\end{table*}


A major
point of difference lies in the value of the specific production rate $\pi$ for
large values of the specific growth rate $\mu$.
Already in the early publications \cite{Falzon87}
it appeared that high glucose
concentrations in the production phase are well correlated with a
low penicillin yield (the 
`glucose effect'). It has been confirmed recently 
\cite{Bunt,Cahour-thesis,BrownAndBurton,Carr-Goldstein}
that
high glucose concentrations inhibit the synthesis of the enzymes of the
penicillin pathway, but not the actual penicillin biosynthesis.
In other words, glucose represses (and not inhibits) the penicillin
biosynthesis. 

These findings do not contradict the results of 
\citeauthor{Chin88-book} (on which \shortcite{Bunt} based their 
production kinetics) and of \citeyear{ChinThesis} which were obtained for
continuous culture fermentations.                
Because for high values of the specific
growth rate $\mu$ it is most likely (as shall be discussed below) that 
maintenance metabolism occurs, it can be shown that
in steady state continuous culture conditions, and with $\mu$ described by a Monod kinetics
\begin{equation}
    C_{s}  =  K_{M} \frac{\mu/\mu_{x}}{1-\mu/\mu_{x}} \label{cs}
\end{equation}
Pirt \& Rhigelato determined $\pi$ for $\mu$ between 
$0.023$ and $0.086$ h$^{-1}$.
They also reported a value $\mu_{x} \approx 0.095$
h$^{-1}$, so that for their experiments $\mu/\mu_{x}$ is in the range 
of $0.24$ to $0.9$. 
Substituting $K _M$ in Eq. (\ref{cs}) by 
the value $K_{M}=1$ g/L as used by \cite{Bunt}, one finds
with the above equation $0.3 < C_{s} < 9$ g/L. This agrees well with 
the work of  \shortcite{Bunt}, who reported that penicillin biosynthesis 
repression only occurs at glucose concentrations from $C_{s}=10$ g/L on.
The conclusion is that the glucose concentrations in the experiments of 
Pirt \& Rhigelato probably were too low for glucose repression to be 
detected. The experimental data published by Ryu \& Hospodka 
are not detailed sufficiently to permit a similar analysis.



\begin{table} 
\caption[]{Parameter sets used by Bajpai \& Reu\ss\ } 
\begin{tabular}{lrll}                                       
\hline
\multicolumn{2}{l}{\it parameter} & {\it Set 1} & {\it Set 2}\\ 
\hline  
$\mu_{x}$           & [h$^{-1}$]  & 0.092       & 0.11          \\
$K_{x}$             & [g/g DM]     & 0.15        & 0.006         \\
$\mu_{p}$           & [g/g DM h]  & 0.005       & 0.004         \\
$K_{p}$             & [g/L]        & 0.0002      & 0.0001        \\
$K_{i}$             & [g/L]        & 0.1         & 0.1           \\
$Y_{x/s}$           & [g DM/g]     & 0.45        & 0.47          \\
$Y_{p/s}$           & [g/g]        & 0.9         & 1.2           \\
$k_{h}$             & [h$^{-1}$]  & 0.04        & 0.01          \\
$m_{s}$             & [g/g DM h]  & 0.014       & 0.029         \\ 
\hline
\end{tabular}
\end{table}
       
Bajpai \& Reu\ss\ decided to disregard the
differences between time constants for the two regulation mechanisms
(glucose repression or inhibition) because of the
relatively very long fermentation times, and therefore proposed a Haldane
expression for $\pi$.

It is interesting that simulations with the \cite{Bunt} model for the
initial conditions given by these authors indicate that, when the
remaining substrate is fed at a constant rate, a considerable and 
unrealistic amount of penicillin is 
produced when the glucose concentration is still very high \cite{CarberryCL88}
Simulations with the Bajpai \& Reu\ss\ model correctly predict almost        
no penicillin production in similar conditions.


The maintenance coefficient used by \cite{Bunt} 
($m_{s}=0.025$ g/g DM h) corresponds well to
the value $m_{s}=0.029$ g/g DM h (Set 2 of \cite{BuchananRBES}), to the 
value $m_{s}=0.024$ g/g DM h reported by \citeauthor{MMI2-d3}, and to the
value used by \shortcite{CawseyIJMMS} ($m_{s}=0.022$ g/g DM h) (1983).
However, these values differ from the value in Set 1 of 
\cite{BuchananRBES} ($m_{s}=0.014$ g/g DM h). 
It is not clear where this difference originated from.
Simulations indicated that the dynamic behaviour of the model is rather
sensitive with respect to the value of $m_{s}$.

In the model van \shortcite{Bunt}, at severe substrate limitation conditions, and thus most probably corresponding
to endogenous metabolic behaviour, the biomass consumption due to maintenance
and production requirements may exceed the conversion of substrate into
biomass and $\mu$ eventually may become negative. This situation may occur at
the end of the growth phase during a fed-batch fermentation.
For these conditions $\pi$ is not defined. A
straightforward extension of the $\pi(\mu)$ kinetics (10)
could be $\pi(\mu \leq 0)=0$, but there
are some biochemical indications that 
the penicillin biosynthesis actually does not stop in that case. 

\begin{figure} % figuur 1
\vspace{6pc}
\caption[]{Pathway of the penicillin G biosynthesis.}
\label{penG}
\end{figure}

Sample of cross-reference to figure.
Figure~\ref{penG} shows that is not easy to get something on paper.



\section{Headings}

\subsection{Subsection}
based their model on balancing methods and
biochemical know\-ledge. The original model (1980) contained an equation for the
oxygen dynamics which has been omitted in a second paper 
(1981). This simplified model shall be discussed here.

\subsubsection{Subsubsection}
\cite{Carr-Goldstein,Cohen-Jones88-book} 
based their model on balancing methods and
biochemical know\-ledge. The original model (1980) contained an equation for the
oxygen dynamics which has been omitted in a second paper 
(1981). This simplified model shall be discussed here.

\paragraph{Paragraph.}
\cite{Carr-Goldstein,Cohen-Jones88-book} 
based their model on balancing methods and
biochemical know\-ledge. The original model (1980) contained an equation for the
oxygen dynamics which has been omitted in a second paper 
(1981). This simplified model shall be discussed here.

\subparagraph{Subparagraph.}
\cite{Carr-Goldstein,Cohen-Jones88-book} 
based their model on balancing methods and
biochemical know\-ledge. The original model (1980) contained an equation for the
oxygen dynamics which has been omitted in a second paper 
(1981). This simplified model shall be discussed here.



\section{Equations and the Like}

Two equations:
\begin{equation}
    C_{s}  =  K_{M} \frac{\mu/\mu_{x}}{1-\mu/\mu_{x}} \label{ccs}
\end{equation}

and

\begin{equation}
    G = \frac{P_{\rm opt} - P_{\rm ref}}{P_{\rm ref}} \mbox{\ }100 \mbox{\ }(\%)
\end{equation}

Two equation arrays:

\begin{eqnarray}
  \frac{dS}{dt} & = & - \sigma X + s_{F} F\\
  \frac{dX}{dt} & = &   \mu    X\\
  \frac{dP}{dt} & = &   \pi    X - k_{h} P\\
  \frac{dV}{dt} & = &   F
\end{eqnarray}

and,

\begin{eqnarray}
 \mu_{\rm substr} & = & \mu_{x} \frac{C_{s}}{K_{x}C_{x}+C_{s}}  \\
 \mu              & = & \mu_{\rm substr} - Y_{x/s}(1-H(C_{s}))(m_{s}+\pi /Y_{p/s}) \\
 \sigma           & = & \mu_{\rm substr}/Y_{x/s}+ H(C_{s}) (m_{s}+ \pi /Y_{p/s})
\end{eqnarray}


\section{References in the THEBIBLIOGRAPHY Environment}
The first reference in the list below has the key \{BrownAndBurton\}.
Together with the bibitem option $\backslash$citeauthoryear\{Brown and
Burton\}\{1978\} the follwing cross-references can be used:\\
$\backslash$cite\{BrownAndBurton\} produces: \cite{BrownAndBurton}\\
$\backslash$shortcite\{BrownAndBurton\}  produces: \shortcite{BrownAndBurton}\\  
$\backslash$citeauthor\{BrownAndBurton\}  produces: \citeauthor{BrownAndBurton}\\  
$\backslash$citeyear\{BrownAndBurton\}  produces: \citeyear{BrownAndBurton}\\  


\appendix

And this is my Appendix.


\acknowledgements
And this is an acknowledgements section with a heading that was produced by the
$\backslash$acknowledgements command. Thank you all for helping me writing this
\LaTeX\ sample file.

% The endnotes section will be placed here.

\theendnotes

\begin{thebibliography}{}

\bibitem[\protect\citeauthoryear{Brown and Burton}{1978}]{BrownAndBurton}
J.~S. Brown and R.~R. Burton.
\newblock {Diagnostic Models for Procedural Bugs in Basic Mathematical Skills}.
\newblock {\em Cognitive Science}, 2(2):155--192, 1978.

\bibitem[\protect\citeauthoryear{Buchanan}{1984}]{BuchananRBES}
Bruce~G. Buchanan and Edward~H. Shortliffe.
\newblock {\em Rule-Based Expert Systems: The MYCIN Experiments of the Stanford
  Heuristic Programming Project}.
\newblock Addison-Wesley Publishing Company, 1984.

\bibitem[\protect\citeauthoryear{Bunt}{1990}]{Bunt}
H.~C. Bunt.
\newblock {Modular Incremental Modelling of Belief and Intention}.
\newblock In {\em Proceedings of the Second International Workshop on User
  Modeling}, 1990.

\bibitem[\protect\citeauthoryear{Cahour}{1990}]{Cahour-90-Montreal}
B\'eatrice Cahour.
\newblock {Competence Modelling in Consultation Dialogs}.
\newblock In L.~Berlinguet and D.~Berthelette, editors, {\em Proceedings of the
  International Congress, Work With Dispay Units' 89}, Montreal, Canada,
  September 1990. North Holland, Amsterdam.

\bibitem[\protect\citeauthoryear{Cahour}{1988}]{Cahour-thesis}
B\'eatrice Cahour.
\newblock {\em {La Mod\'elisation de l'Interlocuteur: Elaboration du Mod\`ele
  et Effets au Cours de Dialogues de Consultation}}.
\newblock PhD thesis, Universit\'e Paris 8, France, 1991.
\newblock Cognitive psychology PhD.

\bibitem[\protect\citeauthoryear{Carberry}{1988}]{CarberryCL88}
Sandra~M. Carberry.
\newblock {Modeling the User's Plans and Goals}.
\newblock {\em Computational Linguistics}, 14(3):23--37, September 1988.

\bibitem[\protect\citeauthoryear{Carbonell}{1970}]{CarbonellScholar}
J.~R. Carbonell.
\newblock {AI in CAI: An Artificial Intelligence Approach to Computer-Aided
  Instruction}.
\newblock {\em IEEE Transactions on Man-Machine Systems}, 11:190--202, 1970.

\bibitem[\protect\citeauthoryear{Carr and Goldstein}{1977}]{Carr-Goldstein}
B.~Carr and I.~Goldstein.
\newblock {Overlays: A Theory of Modelling for Computed Aided Instruction}.
\newblock AI Memo 406, 1977.

\bibitem[\protect\citeauthoryear{Cawsey}{(in press)}]{CawseyIJMMS}
Alison Cawsey.
\newblock {Planning Interactive Explanations}.
\newblock {\em International Journal of Man-Machine Studies}, in press.
                                                               
\bibitem[\protect\citeauthoryear{Chandrasekaran and Swartout}{1991}]{chandra-swartout}
B.~Chandrasekaran and William Swartout.
\newblock {Explanations in Knowledge Systems: The Role of Explicit
  Representation of Design Knowledge}.
\newblock {\em IEEE Expert}, 6(3):47--50, June 1991.

\bibitem[\protect\citeauthoryear{Chappel and Cahour}{1991}]{MMI2-d3}
Helen Chappel and B\'eatrice Cahour.
\newblock {User Modeling for Multi-Modal Co-Operative Dialogue with KBS}.
\newblock Deliverable D3, Esprit Project P2474, 1991.

\bibitem[\protect\citeauthoryear{Chin}{1987}]{ChinThesis}
David~N. Chin.
\newblock {\em Intelligent Agents as a Basis for Natural Language Interfaces}.
\newblock PhD thesis, University of California at Berkeley, 1987.

\bibitem[\protect\citeauthoryear{Chin}{1989}]{Chin88-book}
David~N. Chin.
\newblock {KNOME}: {M}odeling {W}hat the {U}ser {K}nows in {UC}.
\newblock In Alfred Kobsa and Wolfgang Wahlster, editors, {\em User Models in
  Dialog Systems}. Springer-Verlag, Symbolic Computation Series, Berlin
  Heidelberg New York Tokyo, 1989.

\bibitem[\protect\citeauthoryear{Cohen and Jones}{1989}]{Cohen-Jones88-book}
Robin Cohen and Marlene Jones.
\newblock {Incorporating User Models into Expert Systems for Educational
  Diagnosis}.
\newblock In Alfred Kobsa and Wolfgang Wahlster, editors, {\em User Models in
  Dialog Systems}, pages 35 -- 51. Springer-Verlag, Symbolic Computation
  Series, Berlin Heidelberg New York Tokyo, 1989.

\bibitem[\protect\citeauthoryear{Falzon}{1987}]{Falzon87}
Pierre Falzon.
\newblock {Les Dialogues de Diagnostic: L'\'evaluation des Connaissances de
  l'Interlocuteur}.
\newblock Technical Report 747, INRIA, Rocquencourt, France, 1987.

\bibitem[\protect\citeauthoryear{Finin et al.}{1986}]{FininJoshiWebber}
Timothy~W. Finin, Aravind~K. Joshi, and Bonnie~Lynn Webber.
\newblock {Natural Language Interactions with Artificial Experts}.
\newblock {\em Proceedings of the IEEE}, 74(7), July 1986.

\end{thebibliography}
\end{article}
\end{document}
